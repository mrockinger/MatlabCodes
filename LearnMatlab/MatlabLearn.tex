
\documentclass[12pt,a4paper]{article}
%%%%%%%%%%%%%%%%%%%%%%%%%%%%%%%%%%%%%%%%%%%%%%%%%%%%%%%%%%%%%%%%%%%%%%%%%%%%%%%%%%%%%%%%%%%%%%%%%%%%%%%%%%%%%%%%%%%%%%%%%%%%%%%%%%%%%%%%%%%%%%%%%%%%%%%%%%%%%%%%%%%%%%%%%%%%%%%%%%%%%%%%%%%%%%%%%%%%%%%%%%%%%%%%%%%%%%%%%%%%%%%%%%%%%%%%%%%%%%%%%%%%%%%%%%%%
\usepackage{eurosym}
\usepackage{amsmath}

\setcounter{MaxMatrixCols}{10}
%TCIDATA{OutputFilter=LATEX.DLL}
%TCIDATA{Version=5.00.0.2552}
%TCIDATA{<META NAME="SaveForMode" CONTENT="1">}
%TCIDATA{Created=Thursday, December 14, 2000 09:16:34}
%TCIDATA{LastRevised=Sunday, January 16, 2005 11:38:10}
%TCIDATA{<META NAME="GraphicsSave" CONTENT="32">}
%TCIDATA{<META NAME="DocumentShell" CONTENT="Articles\SW\Elbert Walker's">}
%TCIDATA{Language=American English}
%TCIDATA{CSTFile=LaTeX article (bright).cst}

\newtheorem{theorem}{Theorem}
\newtheorem{acknowledgement}[theorem]{Acknowledgement}
\newtheorem{algorithm}[theorem]{Algorithm}
\newtheorem{axiom}[theorem]{Axiom}
\newtheorem{case}[theorem]{Case}
\newtheorem{claim}[theorem]{Claim}
\newtheorem{conclusion}[theorem]{Conclusion}
\newtheorem{condition}[theorem]{Condition}
\newtheorem{conjecture}[theorem]{Conjecture}
\newtheorem{corollary}[theorem]{Corollary}
\newtheorem{criterion}[theorem]{Criterion}
\newtheorem{definition}[theorem]{Definition}
\newtheorem{example}[theorem]{Example}
\newtheorem{exercise}[theorem]{Exercise}
\newtheorem{lemma}[theorem]{Lemma}
\newtheorem{notation}[theorem]{Notation}
\newtheorem{problem}[theorem]{Problem}
\newtheorem{proposition}[theorem]{Proposition}
\newtheorem{remark}[theorem]{Remark}
\newtheorem{solution}[theorem]{Solution}
\newtheorem{summary}[theorem]{Summary}
\newenvironment{proof}[1][Proof]{\textbf{#1.} }{\ \rule{0.5em}{0.5em}}
\def\Pr{\mbox{Pr}}
\def\V{\mbox{V}}
\def\Cov{\mbox{Cov}}
\def\E{\mbox{E}}
\def\C{\mbox{C}}
\def\Var{\mbox{Var}}
\def\R{\underline{R}}
\def\x{\underline{x}}
\def\S{\underline{S}}
\def\1{\mbox{{\bf 1\kern-.24em{I}}}}
\def\R{R}
\def\SE{\mbox{SE}}
\def\Sk{\mbox{Sk}}
\def\Ku{\mbox{Ku}}
\def\Rc{{\cal R}}
\def\G#1#2{\Gamma\left(\frac{#1}{#2}\right)}
\addtolength{\oddsidemargin}{-1cm}
\addtolength{\evensidemargin}{-1cm}
\addtolength{\topmargin}{-2cm}
\addtolength{\textwidth}{2cm}
\addtolength{\textheight}{3cm}
\renewcommand{\baselinestretch}{1.4}
\input{tcilatex}

\begin{document}

\title{Some notes that should help to learn Matlab}
\author{Michael Rockinger}
\maketitle

\section{Introduction}

Computer programs:\ -Matlab/Gauss

\qquad \qquad \qquad \qquad \qquad -Eviews/Rats

\qquad \qquad \qquad \qquad \qquad -S+/R

\qquad \qquad \qquad \qquad \qquad -Stata/Sas

\qquad \qquad \qquad \qquad \qquad -Ad hoc modules - extreme value theory

\qquad \qquad \qquad \qquad \qquad \qquad \qquad \qquad \qquad -Lesage
Econometric toolbox

\qquad \qquad \qquad \qquad \qquad \qquad
http://www.spatial-econometrics.com/

\qquad \qquad \qquad \qquad \qquad - C/C++ (for instance to write DLLs,
Bayesian analysis)

Historical background (Eispack/Linpack - now Lapack)

\section{Matlab}

\subsection{\protect\bigskip Command window}

format long

format short

format bank\qquad \qquad \% two decimals

\bigskip get(0,'Format') \qquad \% object oriented programming

clc

clear

clear name

\bigskip

help

ver

whos

diary

\subsection{Editor}

\subsubsection{basic matrix instructions}

\qquad \qquad

\textbf{learning1.m}

\texttt{\TEXTsymbol{>}\TEXTsymbol{>} }

\texttt{echo on \%also write intstructions on screen}

\texttt{A=[ 1 2 3; 4 5 6] \% creates a Matrix}

\texttt{A =}

\texttt{\ 1.00 2.00 3.00}

\texttt{\ 4.00 5.00 6.00}

\texttt{b=[88; 99]; \% notice the difference due to ;}

\texttt{disp('a column vector');}

\texttt{a column vector}

\texttt{b}

\texttt{b =}

\texttt{\ 88.00}

\texttt{\ 99.00}

\texttt{fprintf('another way to print b but using formatted output. b=\%12.4f%
\TEXTsymbol{\backslash}n',b);}

\texttt{another way to print b but using formatted output. b= 88.0000}

\texttt{another way to print b but using formatted output. b= 99.0000}

\texttt{disp('what is the precision of Matlab');}

\texttt{what is the precision of Matlab}

\texttt{fprintf('pi=\%25.20f',pi);}

\texttt{pi= 3.14159265358979310000}

\texttt{A(:,1) \% the first column of A}

\texttt{ans =}

\texttt{\ 1.00}

\texttt{\ 4.00}

\texttt{A(2,:) \% the second row of A}

\texttt{ans =}

\texttt{\ 4.00 5.00 6.00}

\texttt{A(1,3) \% an element out of A}

\texttt{ans =}

\texttt{\ 3.00}

\texttt{A(2,2:3)=[-2 -3]; \% replace blocks in A}

\texttt{A}

\texttt{A =}

\texttt{\ 1.00 2.00 3.00}

\texttt{\ 4.00 -2.00 -3.00}

\texttt{B=[1 7 3 ; 5 6 7] \% another matrix}

\texttt{B =}

\texttt{\ 1.00 7.00 3.00}

\texttt{\ 5.00 6.00 7.00}

\texttt{B' \%transpose of B}

\texttt{ans =}

\texttt{\ 1.00 5.00}

\texttt{\ 7.00 6.00}

\texttt{\ 3.00 7.00}

\texttt{C=B'*A \%yet another matrix}

\texttt{C =}

\texttt{\ 21.00 -8.00 -12.00}

\texttt{\ 31.00 2.00 3.00}

\texttt{\ 31.00 -8.00 -12.00}

\texttt{5*C \% multiply a matrix by a scalar}

\texttt{ans =}

\texttt{\ 105.00 -40.00 -60.00}

\texttt{\ 155.00 10.00 15.00}

\texttt{\ 155.00 -40.00 -60.00}

\texttt{D=[A;B] \% vertical concatenation (combine vertically A and B)}

\texttt{D =}

\texttt{\ 1.00 2.00 3.00}

\texttt{\ 4.00 -2.00 -3.00}

\texttt{\ 1.00 7.00 3.00}

\texttt{\ 5.00 6.00 7.00}

\texttt{D=[A B] \% horizontal concatenation}

\texttt{D =}

\texttt{\ Columns 1 through 4 }

\texttt{\ 1.00 2.00 3.00 1.00}

\texttt{\ 4.00 -2.00 -3.00 5.00}

\texttt{\ Columns 5 through 6 }

\texttt{\ 7.00 3.00}

\texttt{\ 6.00 7.00}

\texttt{A.*B \% Haddamar product (element by element product)}

\texttt{ans =}

\texttt{\ 1.00 14.00 9.00}

\texttt{\ 20.00 -12.00 -21.00}

\texttt{E=[] \% an empty matrix}

\texttt{E =}

\texttt{\ []}

\texttt{E=[E;A;A] \%concatenate matrices (useful to store results in E)}

\texttt{E =}

\texttt{\ 1.00 2.00 3.00}

\texttt{\ 4.00 -2.00 -3.00}

\texttt{\ 1.00 2.00 3.00}

\texttt{\ 4.00 -2.00 -3.00}

\texttt{A+B \%sum of matrices}

\texttt{ans =}

\texttt{\ 2.00 9.00 6.00}

\texttt{\ 9.00 4.00 4.00}

\texttt{A-B \% difference of matrices}

\texttt{ans =}

\texttt{\ 0 -5.00 0}

\texttt{\ -1.00 -8.00 -10.00}

\texttt{a=A(1,:)}

\texttt{a =}

\texttt{\ 1.00 2.00 3.00}

\texttt{b=B(:,2)}

\texttt{b =}

\texttt{\ 7.00}

\texttt{\ 6.00}

\texttt{[r,c]=size(A) \%rows and columns of A}

\texttt{r =}

\texttt{\ 2.00}

\texttt{c =}

\texttt{\ 3.00}

\texttt{size(A,1) \% row dimension of A}

\texttt{ans =}

\texttt{\ 2.00}

\texttt{size(A,2) \% column dimension of A}

\texttt{ans =}

\texttt{\ 3.00}

\texttt{\% notice: a.*b would return a mistake}

\texttt{kron(a,b)}

\texttt{ans =}

\texttt{\ 7.00 14.00 21.00}

\texttt{\ 6.00 12.00 18.00}

\texttt{clear a}

\texttt{\%just typing a returns an error}

\texttt{A=1:3 \% creates a sequence for A. Notice A is a row!!!}

\texttt{A =}

\texttt{\ 1.00 2.00 3.00}

\texttt{A(end)}

\texttt{ans =}

\texttt{\ 3.00}

\texttt{A(end-1)}

\texttt{ans =}

\texttt{\ 2.00}

\texttt{B='I am a string of characters'}

\texttt{B =}

\texttt{I am a string of characters}

\texttt{C='; lots of them'}

\texttt{C =}

\texttt{; lots of them}

\texttt{D=[B C] \% horizontal concatenation}

\texttt{D =}

\texttt{I am a string of characters; lots of them}

\texttt{S=char(128) \% the Euro}

\texttt{S =}

\texttt{\euro }

\texttt{D=[D ' ' S] }

\texttt{D =}

\texttt{I am a string of characters; lots of them \euro }

\texttt{E=strvcat(B,'Yes') \% vertical concatenation. Notice padding}

\texttt{E =}

\texttt{I am a string of characters}

\texttt{Yes }

\texttt{disp(E)}

\texttt{I am a string of characters}

\texttt{Yes }

\texttt{E=strcat(B,' No') \% equivalent to [B ' No']. Emphasis on string
operation}

\texttt{E =}

\texttt{I am a string of characters No}

\texttt{double('A') \% to go from char to integer}

\texttt{ans =}

\texttt{\ 65.00}

\texttt{double('a')}

\texttt{ans =}

\texttt{\ 97.00}

\texttt{x=randn(3,1) \% generates normal random numbers}

\texttt{x =}

\texttt{\ 0.53}

\texttt{\ 0.22}

\texttt{\ -0.92}

\texttt{str1 = num2str(min(x)) \% a first string}

\texttt{str1 =}

\texttt{-0.9219}

\texttt{str2 = num2str(max(x)) \% another one}

\texttt{str2 =}

\texttt{0.52874}

\texttt{disp(str1); \%only strings can get displayed with disp}

\texttt{-0.9219}

\texttt{['from smallest ' str1 ' to largest ' str2] }

\texttt{ans =}

\texttt{from smallest -0.9219 to largest 0.52874}

\texttt{s=num2str(min(x),12) \% more decimals}

\texttt{s =}

\texttt{-0.921901624356}

\texttt{['now display 12 decimals ',s]}

\texttt{ans =}

\texttt{now display 12 decimals -0.921901624356}

\texttt{str = '12.3389e-1';}

\texttt{val = str2num(str) \% goes from string to real}

\texttt{val =}

\texttt{\ 1.23}

\texttt{val+12}

\texttt{ans =}

\texttt{\ 13.23}

\texttt{\% do not confuse with cells. These are structures.}

\texttt{\% cells can be useful for collecting output from a function}

\texttt{A=\{[1; 2] 'I scream for ice cream' \}}

\texttt{A = }

\texttt{\ [2x1 double] [1x22 char]}

\texttt{A\{1,1\}}

\texttt{ans =}

\texttt{\ 1.00}

\texttt{\ 2.00}

\texttt{A\{1,2\}}

\texttt{ans =}

\texttt{I scream for ice cream}

\texttt{\TEXTsymbol{>}\TEXTsymbol{>} }

Comments: \%

Lines that are too: long use three points and continue on the next line ...

\bigskip

\subsubsection{some math functions}

\textbf{learning2.m}

\texttt{A=[1 2 3; 1 2 3]}

\texttt{A =}

\texttt{\ 1 2 3}

\texttt{\ 1 2 3}

\texttt{rank(A) \%rank of A}

\texttt{ans =}

\texttt{\ 1}

\texttt{B=[1 2; 3 4]}

\texttt{B =}

\texttt{\ 1 2}

\texttt{\ 3 4}

\texttt{rank(B)}

\texttt{ans =}

\texttt{\ 2}

\texttt{det(B) \%determinant}

\texttt{ans =}

\texttt{\ -2}

\texttt{trace(B) \%trace}

\texttt{ans =}

\texttt{\ 5}

\texttt{[V,D] = eig(B) \%eigen vectors and eigen values}

\texttt{V =}

\texttt{\ -0.82456484013239 -0.41597355791928}

\texttt{\ 0.56576746496899 -0.90937670913212}

\texttt{D =}

\texttt{\ -0.37228132326901 0}

\texttt{\ 0 5.37228132326901}

\texttt{[B*V(:,1) D(1,1)*V(:,1)] \%verification that B v\_1= lda\_1 v\_1}

\texttt{ans =}

\texttt{\ 0.30697008980559 0.30697008980559}

\texttt{\ -0.21062466052121 -0.21062466052121}

\texttt{inv(B)}

\texttt{ans =}

\texttt{\ -2.00000000000000 1.00000000000000}

\texttt{\ 1.50000000000000 -0.50000000000000}

\texttt{c=[3; 4]}

\texttt{c =}

\texttt{\ 3}

\texttt{\ 4}

\texttt{inv(B)*c \% one way to compute}

\texttt{ans =}

\texttt{\ -2.00000000000000}

\texttt{\ 2.50000000000000}

\texttt{\% a much better way to go (for numerical precision)}

\texttt{B\TEXTsymbol{\backslash}c}

\texttt{ans =}

\texttt{\ -2.00000000000000}

\texttt{\ 2.50000000000000}

\texttt{E=eye(2)}

\texttt{E =}

\texttt{\ 1 0}

\texttt{\ 0 1}

\texttt{\% if one needs to invert B use (faster + more precise)}

\texttt{B\TEXTsymbol{\backslash}E}

\texttt{ans =}

\texttt{\ -2.00000000000000 1.00000000000000}

\texttt{\ 1.50000000000000 -0.50000000000000}

\texttt{\% condition number (ratio of largest to smallest eigenvalue)}

\texttt{\% if larger than 30'000 be careful when invering matrix}

\texttt{cond(B)}

\texttt{ans =}

\texttt{\ 14.93303437365927}

\texttt{C=[1 2; 0.9999 2.0001]}

\texttt{C =}

\texttt{\ 1.00000000000000 2.00000000000000}

\texttt{\ 0.99990000000000 2.00010000000000}

\texttt{cond(C)}

\texttt{ans =}

\texttt{\ 3.333400003665147e+004}

\texttt{\%Choleski decomposition matrix must be positive definite}

\texttt{B=[1 0.1; 0.1 3]}

\texttt{B =}

\texttt{\ 1.00000000000000 0.10000000000000}

\texttt{\ 0.10000000000000 3.00000000000000}

\texttt{X=chol(B)}

\texttt{X =}

\texttt{\ 1.00000000000000 0.10000000000000}

\texttt{\ 0 1.72916164657906}

\texttt{X'*X \%verification}

\texttt{ans =}

\texttt{\ 1.00000000000000 0.10000000000000}

\texttt{\ 0.10000000000000 3.00000000000000}

\bigskip

The usual set of trigonometric functions works

sin, cos, tan, cot, sinh, cosh, tanh, coth, asin, acos, atan, acot, and
arc-hyperbolic functions.

Further functions

\bigskip 
\begin{tabular}{ll|ll}
name & instruction & name & instruction \\ 
Exponential & exp & Round towards plus infinity & ceil \\ 
Natural logarithm & log & Round towards nearest integer & round \\ 
Base 2 logarithm & log2 & Modulus & mod \\ 
(base 10) logarithm & log10 & Signum & sign \\ 
Square root & sqrt & Prime factors & factor \\ 
Absolute value & abs & Factorial function & factorial \\ 
Complex conjugate & conj & Greatest common divisor & gcd \\ 
Complex number & complex & True for prime numbers & isprime \\ 
Phase angle & angle & Least common multiple & lcm \\ 
Imaginary unit & i ou j & Choose k among N & nchoosek \\ 
Complex imaginary part & imag & All possible permutations & perms \\ 
True for real array & isreal & Generate list of prime numbers & primes \\ 
Complex real part & real &  &  \\ 
Round towards zero & fix &  &  \\ 
Round towards minus infinity & floor &  & 
\end{tabular}

\bigskip Some generally useful functions (also for elementary statistics)

\begin{tabular}{ll|ll}
name & instruction & name & instruction \\ 
Cumulative product & cumprod & Product of array elts. & prod \\ 
Cumulative sum & cumsum & Sort elements in ascd. order & sort \\ 
Cum. trapezoidal num. integration & cumtrapz & Sort rows in ascd. order & 
sortrows \\ 
Maximum elt. of array (also index) & max & Standard deviation & std \\ 
Average = mean of arrays & mean & Sum of array elements & sum \\ 
Median value of arrays & median & Trapezoidal num. integ. & trapz \\ 
Minimum elt. of array (also index) & min & Variance & var \\ 
Differences & diff & Numerical gradient & gradient \\ 
Correlation coefficients & Correlation & Covariance matrix & cov%
\end{tabular}

\subsubsection{Mathematical operators}

+ \ \ - \ * \ / \ \ \ \ \%addition, subtraction, multiplication, division

' \ \ \% \ ! \ \ \symbol{94}\ \ \ \ \%transpose, moduls division, factional,
exponentiation

\subsubsection{Relational operators}

\begin{tabular}{cccccc}
== & \symbol{126}= & \TEXTsymbol{>} & \TEXTsymbol{<} & \TEXTsymbol{>}= & 
\TEXTsymbol{<}= \\ 
equal & 
\begin{tabular}{c}
not \\ 
equal%
\end{tabular}
& 
\begin{tabular}{c}
greater \\ 
than%
\end{tabular}
& 
\begin{tabular}{c}
less \\ 
than%
\end{tabular}
& 
\begin{tabular}{c}
greater or \\ 
equal%
\end{tabular}
& 
\begin{tabular}{c}
less or \\ 
equal%
\end{tabular}%
\end{tabular}

A=5 \ is not the same as A==5;

\subsubsection{Logical operators}

\& AND

\TEXTsymbol{\vert} OR

\symbol{126} NON

xor

Sometimes a true condition corresponds to 1. Matlab\ does not always adhere
to this because of rounding error.

\subsection{Programming}

Edit a filename: Ex. gmm1.m

good habit$\rightarrow \qquad $\ *.m\qquad \qquad program file\qquad

\qquad \qquad \qquad \qquad *.txt \ \ \ \ \ \ \ \qquad data file

\qquad \qquad \qquad \qquad *.dat\qquad \qquad data file

\qquad \qquad \qquad \qquad *.asc\qquad \qquad data file

\qquad \qquad \qquad \qquad *.src \qquad\ \ \ \ \ \ source codes

\bigskip

further good habits$\rightarrow \qquad \ \ $i,j,k,l $\rightarrow $ index
variables

\qquad \qquad \qquad \qquad \qquad \qquad small $\rightarrow $letters scalar

\qquad \qquad \qquad \qquad \qquad \qquad capital $\rightarrow $ letters
matrices

notice that a and A are not the same thing for Matlab

\_A \ \ \ \ underscore\ as a first letter is not allowed

3A not allowed. A3=2. This is allowed

\bigskip

\subsubsection{A first 'official' program to illustrate input/output}

Create a database in Excel. Save it as an ascii file my2.txt. Use tabulation
as operator.

{\small \qquad \qquad 1\qquad 6.7\qquad 2.33}

{\small \qquad \qquad 2\qquad 8.9\qquad 1.25}

{\small \qquad \qquad 3 \qquad 3.3\qquad 8.7}

{\small \qquad \qquad 4\qquad 8.7\qquad 9.3}

{\small \qquad \qquad 5\qquad 2.3 \ \ \ \ \ 10.33}

{\small \qquad \qquad 6\qquad 2.99 \ \ \ 15}

\bigskip

Matlab is as good (or as bad) as other programs at reading files. If you
have an Excel file, then the easiest is to use something like

\texttt{A = xlsread('filename').}

\bigskip The following program performs some basic input/output.

\texttt{\ function learning3()}

\texttt{\ \% learning3.m}

\texttt{\ \% a first matlab function}

\texttt{\ \% loads some ascii data, }

\texttt{\ \% creates with that data a Matlab dataset. Loads data and
displays it}

\texttt{\ \%}

\texttt{\ clc;}

\texttt{\ fid = fopen('d:\TEXTsymbol{\backslash}\TEXTsymbol{\backslash}rocky%
\TEXTsymbol{\backslash}\TEXTsymbol{\backslash}my2.txt','r')}

\texttt{\ [A,T] = fscanf(fid,'\%f \%f \%f'); \% notice: no missing values
allowed}

\texttt{\ fclose(fid);}

\texttt{\ }

\texttt{\ format short;}

\texttt{\ A(1:3)}

\texttt{\ A(end-2:end)}

\texttt{\ disp('T');}

\texttt{\ T}

\texttt{\ x=reshape(A,3,T/3)' \%create 3 rows then transpose Each unit has
T/3 elements}

\texttt{\ x}

\texttt{\ }

\texttt{\ \% now writes data into a new file}

\texttt{\ disp('Save data into file');}

\texttt{\ save 'd:\TEXTsymbol{\backslash}finbox\TEXTsymbol{\backslash}my2' x;%
}

\texttt{\ whos}

\texttt{\ clear;}

\texttt{\ disp('whos after clear')}

\texttt{\ whos}

\texttt{\ load 'd:\TEXTsymbol{\backslash}finbox\TEXTsymbol{\backslash}my2' x}

\texttt{\ x}

\texttt{\ }

\texttt{\ \% save as an ascii file }

\texttt{\ fid = fopen('d:\TEXTsymbol{\backslash}\TEXTsymbol{\backslash}finbox%
\TEXTsymbol{\backslash}\TEXTsymbol{\backslash}my2.txt','w');}

\texttt{\ fprintf(fid,'\%5.0f \%8.3f \%8.3f \TEXTsymbol{\backslash}n',x);
\%don't forget the \TEXTsymbol{\backslash}n}

\texttt{\ \% notice \TEXTsymbol{\backslash}n is readable under word but not
under wordpad...}

\texttt{\ fclose(fid);}

\bigskip

Running this file produces

\texttt{fid =}

\texttt{\ 3}

\texttt{ans =}

\texttt{\ 1.0000}

\texttt{\ 6.7000}

\texttt{\ 2.3300}

\texttt{ans =}

\texttt{\ 6.0000}

\texttt{\ 2.9900}

\texttt{\ 15.0000}

\texttt{T}

\texttt{T =}

\texttt{\ 18}

\texttt{x =}

\texttt{\ 1.0000 6.7000 2.3300}

\texttt{\ 2.0000 8.9000 1.2500}

\texttt{\ 3.0000 3.3000 8.7000}

\texttt{\ 4.0000 8.7000 9.3000}

\texttt{\ 5.0000 2.3000 10.3000}

\texttt{\ 6.0000 2.9900 15.0000}

\texttt{Save data into file}

\texttt{\ Name Size Bytes Class}

\texttt{\ A 18x1 144 double array}

\texttt{\ T 1x1 8 double array}

\texttt{\ ans 3x1 24 double array}

\texttt{\ fid 1x1 8 double array}

\texttt{\ x 6x3 144 double array}

\texttt{Grand total is 41 elements using 328 bytes}

\bigskip

\texttt{whos after clear}

\bigskip

\texttt{x =}

\texttt{\ 1.0000 6.7000 2.3300}

\texttt{\ 2.0000 8.9000 1.2500}

\texttt{\ 3.0000 3.3000 8.7000}

\texttt{\ 4.0000 8.7000 9.3000}

\texttt{\ 5.0000 2.3000 10.3000}

\texttt{\ 6.0000 2.9900 15.0000}

\subsubsection{Conditional branching}

\bigskip

\texttt{for m = 1:k}

\qquad \texttt{for n = 1:k}

\qquad \qquad \texttt{if m == n}

\qquad \qquad \qquad \texttt{a(m,n) = 2;}

\qquad \qquad \texttt{elseif abs(m-n) == 2}

\qquad \qquad \qquad \texttt{a(m,n) = 1;}

\qquad \qquad \texttt{else}

\qquad \qquad \qquad \texttt{a(m,n) = 0;}

\texttt{\ \qquad \qquad end}

\texttt{\ \qquad end}

\texttt{end}

For k=5 you get the matrix

\texttt{a =}

\texttt{\ 2 0 1 0 0}

\texttt{\ 0 2 0 1 0}

\texttt{\ 1 0 2 0 1}

\texttt{\ 0 1 0 2 0}

\texttt{\ 0 0 1 0 2}

\subsubsection{A really small test}

If condition

\qquad do something;

end

notice the (non)-use of ;

Tests may be nested.

\subsubsection{While/Until loops}

while condition

\qquad do something;

end

Illustration

\textbf{WhileEx.m}

\texttt{i=1;}

\texttt{while i\TEXTsymbol{<}10}

\texttt{\ [i i\symbol{94}2]}

\texttt{\ i=i+1;}

\texttt{end}

\texttt{Res=[];}

\texttt{A=zeros(10,1);}

\texttt{for x=1:0.1:2}

\texttt{\ Res=[Res; [x x\symbol{94}2]]; \% only ok for short series else
pre-declare size}

\texttt{\ A(floor(x*10),:)=x}

\texttt{end}

\texttt{Res}

\texttt{disp('A');}

\texttt{A}

\bigskip

Remark: Until (of some other languages) can always be recoded as while$%
\Leftrightarrow $ Matlab only implements while.

The loop is tested before it is executed $\rightarrow $\ loop may not be
executed at all.

if condition depends on something affected within loop, the programmer must
ensure that he does not forget ''something''. This is a typical source of
error getting your program to run for ever.

\subsubsection{For loops}

\textbf{loops.m}

\texttt{k=3; Res=[];}

\texttt{for m = 1:k}

\texttt{\qquad Res=[Res;[m m\symbol{94}2]];}

\texttt{end}

produces

\texttt{Res =}

\texttt{\ 1 1}

\texttt{\ 2 4}

\texttt{\ 3 9}

$\rightarrow $It's much faster than while

$\rightarrow $Requires that number of steps are known \ \ \ 

The instruction break may be used to get out of a for or a while loop.\ \ \ 

\subsubsection{Function evaluations}

involvs \textbf{fcall.m} and \textbf{do\_more\_Comput.m}

\texttt{function fcall()}

\texttt{\ \% illustrates how to nest functions and local/globalness of
variables}

\texttt{\ global c}

\texttt{\ a=2}

\texttt{\ b=3}

\texttt{\ c=10}

\texttt{\ }

\texttt{\ [x,y]=do\_Comput(a,b) \%here involves two elements that are
returned}

\texttt{\ }

\texttt{\ function [f,g]=do\_Comput(a,b);}

\texttt{\ global c}

\texttt{\ f=a\symbol{94}2+c; }

\texttt{\ g=a*b*c;}

\bigskip

this produces

\texttt{a =}

\texttt{\ 2}

\texttt{b =}

\texttt{\ 3}

\texttt{c =}

\texttt{\ 10}

\texttt{x =}

\texttt{\ 14}

\texttt{y =}

\texttt{\ 60}

Sometimes you wish to develop a general program such as an optimizer (see a
bit later) or a module that is very general and that you wish to run on many
functions (think about a Max Lik program). In that case, you have to provide
the name of a function. This can be achieved via \texttt{@function\_name} at
the level of the calling program and via \texttt{feval(function\_name)} at
the function level.

\textbf{showeval.m}

\texttt{function showeval()}

\texttt{\% illustrates how 'feval' operates}

\texttt{\% here compute gradients for two trivial functions at trivial points%
}

\texttt{x=1.0;}

\texttt{grad(@f,x) \% evaluates grandient of f at x}

\texttt{grad(@g,x)}

\texttt{function y=f(x);}

\texttt{y=x\symbol{94}2;}

\texttt{function y=g(x)}

\texttt{y=x\symbol{94}3;}

\texttt{function gr=grad(fnam,x)}

\texttt{h=0.0000001;}

\texttt{f1=feval(fnam,x);}

\texttt{f2=feval(fnam,x+h);}

\texttt{gr=(f2-f1)/h}

\subsection{\protect\bigskip Using the optimizer}

\bigskip \textbf{Optim\_Ilustrate.m}

\texttt{function Optim\_Ilustrate()}

\texttt{\% illustrates how to use the optimization package}

\texttt{clc;}

\texttt{options = optimset('MaxFunEvals', 1000, ...}

\texttt{'TolFun', 1e-5,...}

\texttt{\ 'TolX', 1e-5,...}

\texttt{'LargeScale','off',...}

\texttt{'Diagnostics','on',...}

\texttt{'Display','iter');}

\texttt{beta0=[1 2]; \% starting values}

\texttt{A=3;}

\texttt{[beta,Qmin,exitflag] = fminunc(@obj\_func,beta0,options,A);}

\texttt{disp('the optimum is');}

\texttt{beta}

\texttt{disp('whereas it should be ');}

\texttt{[A;-1]}

\texttt{function z=obj\_func(beta,A);}

\texttt{x=beta(1);}

\texttt{y=beta(2);}

\texttt{z=(x-A)\symbol{94}2+(y+1)\symbol{94}2;}

running this program produces:

\texttt{\TEXTsymbol{>} In C:\TEXTsymbol{\backslash}MATLAB6p1\TEXTsymbol{%
\backslash}toolbox\TEXTsymbol{\backslash}optim\TEXTsymbol{\backslash}private%
\TEXTsymbol{\backslash}diagnose.m at line 20}

\texttt{\ In C:\TEXTsymbol{\backslash}MATLAB6p1\TEXTsymbol{\backslash}toolbox%
\TEXTsymbol{\backslash}optim\TEXTsymbol{\backslash}fminunc.m at line 221}

\texttt{\ In D:\TEXTsymbol{\backslash}finbox\TEXTsymbol{\backslash}%
Learning4.m at line 12}

\texttt{Number of variables: 2}

\texttt{Functions }

\texttt{\ Objective: obj\_func}

\texttt{\ Gradient: finite-differencing}

\texttt{\ Hessian: finite-differencing (or Quasi-Newton)}

\texttt{\ }

\texttt{Algorithm selected}

\texttt{\ medium-scale: Quasi-Newton line search}

\texttt{\%\%\%\%\%\%\%\%\%\%\%\%\%\%\%\%\%\%\%\%\%\%\%\%\%\%\%\%\%\%\%\%\%\%%
\%\%\%\%\%\%\%\%\%\%\%\%\%\%\%\%\%\%\%\%\%\%\%\%\%}

\texttt{\ End diagnostic information }

\texttt{\ Directional }

\texttt{\ Iteration Func-count f(x) Step-size derivative }

\texttt{\ 1 2 13 0.5 -52 }

\texttt{\ 2 8 3.22214e-017 0.5 2.62e-009 }

\texttt{\ }

\texttt{Optimization terminated successfully:}

\texttt{\ Search direction less than 2*options.TolX}

\texttt{the optimum is}

\texttt{beta =}

\texttt{\ 3.0000 -1.0000}

\texttt{whereas it should be }

\texttt{ans =}

\texttt{\ 3}

\texttt{\ -1}

\texttt{\bigskip }

Notice: a function can only be nested within a function

\textbf{Compilation mcc -x filename}

\subsubsection{Efficient programming}

Matlab procedures which are rationalized exist for an incredible number of
situations.

\qquad - these procedures are pre-tested. Keeps your programming fast

\qquad - procedures are vectorized. They are much faster than loops for
instance.

\qquad - Time your loops using tic toc

\qquad - Always put the largest loop as most innermost\ loop.

\qquad - Declare repeated operations outside loop (for instance if you need
to compute a loop involving i*gamma(3) where i is some increment, then it
would be too costly to evaluate in each loop the gamma function. Compute out
of the loop g=gamma(3) and use in the loop i*g.

Declare place holders rather than using concatenation. For instance use

\texttt{nh=n/2;}

\texttt{np1h=(n+1)/2;}

\texttt{y=zeros(T,1);}

\texttt{j=1;}

\texttt{while j\TEXTsymbol{<}=T \% notice, here there is not need for a ;}

\texttt{\qquad y(j)=gamma(nh\symbol{94}j)/gamma(np1h\symbol{94}j)*x(j); \% x
defined elsewhere }

\texttt{\qquad j=j+1;}

\texttt{end \%again no need for a ;}

which is much faster than

\texttt{j=1;}

\texttt{y=[];}

\texttt{while j\TEXTsymbol{<}=T}

\texttt{\qquad y=[y; (gamma(n/2)\symbol{94}j)/gamma(((n+1)/2)\symbol{94}%
j)*x(j)]; \qquad }

\texttt{\qquad j=j+1;}

\texttt{end}

Other example

\textbf{speed.m}

\section{The graphics module}

\subsection{\protect\bigskip Basic graphs (and more general programming)}

The following program is contained in \textbf{ElemStats.m}

\texttt{\TEXTsymbol{>}\TEXTsymbol{>} }

\texttt{echo on}

\texttt{u=randn(5,2) \% 5 uniform random numbers in two columns}

\texttt{u =}

\texttt{\ -0.0132 1.7701}

\texttt{\ -0.5803 0.3255}

\texttt{\ 2.1363 -1.1190}

\texttt{\ -0.2576 0.6204}

\texttt{\ -1.4095 1.2698}

\texttt{u(:,2)=u(:,1)+0.3*u(:,2)}

\texttt{u =}

\texttt{\ -0.0132 0.5179}

\texttt{\ -0.5803 -0.4826}

\texttt{\ 2.1363 1.8006}

\texttt{\ -0.2576 -0.0715}

\texttt{\ -1.4095 -1.0286}

\texttt{plot(u(:,1),u(:,2),'s',...}

\texttt{\ 'MarkerEdgeColor','k',...}

\texttt{\ 'MarkerFaceColor','g',...}

\texttt{\ 'MarkerSize',10)}

\texttt{title('scatterplot of u(:,1) against u(:,2)');}

\texttt{\% in the plot instruction 's' means symbols="Marker"}

\texttt{[ma,ma\_idx]=max(u)}

\texttt{ma =}

\texttt{\ 2.1363 1.8006}

\texttt{ma\_idx =}

\texttt{\ 3 3}

\texttt{max(u) \% notice that Matlab functions are overloaded}

\texttt{ans =}

\texttt{\ 2.1363 1.8006}

\texttt{[mi,mi\_idx]=min(u)}

\texttt{mi =}

\texttt{\ -1.4095 -1.0286}

\texttt{mi\_idx =}

\texttt{\ 5 5}

\texttt{mu=mean(u)}

\texttt{mu =}

\texttt{\ -0.0249 0.1471}

\texttt{sum(u)/size(u,1)}

\texttt{ans =}

\texttt{\ -0.0249 0.1471}

\texttt{median(u)}

\texttt{ans =}

\texttt{\ -0.2576 -0.0715}

\texttt{sort(u)}

\texttt{ans =}

\texttt{\ -1.4095 -1.0286}

\texttt{\ -0.5803 -0.4826}

\texttt{\ -0.2576 -0.0715}

\texttt{\ -0.0132 0.5179}

\texttt{\ 2.1363 1.8006}

\texttt{sortrows(u)}

\texttt{ans =}

\texttt{\ -1.4095 -1.0286}

\texttt{\ -0.5803 -0.4826}

\texttt{\ -0.2576 -0.0715}

\texttt{\ -0.0132 0.5179}

\texttt{\ 2.1363 1.8006}

\texttt{uu = u - kron( mu, ones(size(u,1),1)); \%this is truly stupid}

\texttt{mean( uu.\symbol{94}2 )}

\texttt{ans =}

\texttt{\ 1.3901 0.9396}

\texttt{T=size(u,1);}

\texttt{mean(uu.\symbol{94}2)*(T-1)/T}

\texttt{ans =}

\texttt{\ 1.1121 0.7517}

\texttt{v=var(u)}

\texttt{v =}

\texttt{\ 1.7377 1.1745}

\texttt{sqrt(v)}

\texttt{ans =}

\texttt{\ 1.3182 1.0838}

\texttt{std(u)}

\texttt{ans =}

\texttt{\ 1.3182 1.0838}

\texttt{mu=0.1;}

\texttt{sig=0.2;}

\texttt{S0=100;}

\texttt{x=mu+sig*u;}

\texttt{x1=exp(x);}

\texttt{x1=[ones(1,2); x1]; \% first observation}

\texttt{St1=100*cumprod(x1)}

\texttt{St1 =}

\texttt{\ 100.0000 100.0000}

\texttt{\ 110.2262 122.5772}

\texttt{\ 108.4709 123.0045}

\texttt{\ 183.7808 194.8715}

\texttt{\ 192.9093 212.3079}

\texttt{\ 160.8246 191.0087}

\texttt{x=[zeros(1,2); x]; \% again for first observation}

\texttt{St2=100*exp(cumsum(x))}

\texttt{St2 =}

\texttt{\ 100.0000 100.0000}

\texttt{\ 110.2262 122.5772}

\texttt{\ 108.4709 123.0045}

\texttt{\ 183.7808 194.8715}

\texttt{\ 192.9093 212.3079}

\texttt{\ 160.8246 191.0087}

\texttt{A=[1 2 3; 2 2 4; 2 1 2]}

\texttt{A =}

\texttt{\ 1 2 3}

\texttt{\ 2 2 4}

\texttt{\ 2 1 2}

\texttt{prod(A)}

\texttt{ans =}

\texttt{\ 4 4 24}

\texttt{cumprod(A)}

\texttt{ans =}

\texttt{\ 1 2 3}

\texttt{\ 2 4 12}

\texttt{\ 4 4 24}

\texttt{dx=0.1}

\texttt{dx =}

\texttt{\ 0.1000}

\texttt{x=-5:dx:5; \% some fine grid}

\texttt{y=1/sqrt(2*pi)*exp(-0.5*x.\symbol{94}2);}

\texttt{figure(1)}

\texttt{subplot(2,1,1)}

\texttt{plot(x,y);}

\texttt{cdf=cumtrapz(y)*dx;}

\texttt{subplot(2,1,2)}

\texttt{plot(x,cdf);}

\texttt{trapz(y)*dx}

\texttt{ans =}

\texttt{\ 1.0000}

\FRAME{ftbpF}{5.348in}{4.0171in}{0pt}{}{}{Figure}{\special{language
"Scientific Word";type "GRAPHIC";maintain-aspect-ratio TRUE;display
"USEDEF";valid_file "T";width 5.348in;height 4.0171in;depth
0pt;original-width 5.2918in;original-height 3.9686in;cropleft "0";croptop
"1";cropright "1";cropbottom "0";tempfilename
'I31W8Z00.bmp';tempfile-properties "XPR";}}

\subsection{Another graphical example}

\texttt{\%\textbf{Graphcapa}.m}

\texttt{\% illustrates graphical capabilities}

\texttt{x=randn(10,1);}

\texttt{plot(x,'--+'); \% plots x against time}

\texttt{axis([0 13 -3 3]);}

\texttt{title('a nice simulation')}

\texttt{xlabel('time goes by')}

\texttt{ylabel('the realizations')}

\texttt{legend('name of curve')}

\texttt{text(8,-2,'Right lower corner');}

\subsection{\protect\bigskip More advanced issues}

clc clears command window

clf clears current graphics window

hold on / hold off to superpose plots

subplot(m,n,p) splits graphics window into m rows and n celles. p
corresponds to the number of subplot.

how to glue a figure into word-processor

\bigskip

Object oriented stuff: get/set

get try \texttt{get(0)} in command window

\bigskip

\texttt{\% \textbf{plot2.m} again a plot}

\texttt{\%}

\texttt{x=randn(10,1);}

\texttt{y=1+0.2*randn(10,1);}

\texttt{plot(x,y)}

\texttt{get(gcf,'Position') \%gcf is generic graphics window handle}

\end{document}
