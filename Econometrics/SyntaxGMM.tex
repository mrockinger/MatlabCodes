
\documentclass[12pt,a4paper]{article}
%%%%%%%%%%%%%%%%%%%%%%%%%%%%%%%%%%%%%%%%%%%%%%%%%%%%%%%%%%%%%%%%%%%%%%%%%%%%%%%%%%%%%%%%%%%%%%%%%%%%%%%%%%%%%%%%%%%%%%%%%%%%%%%%%%%%%%%%%%%%%%%%%%%%%%%%%%%%%%%%%%%%%%%%%%%%%%%%%%%%%%%%%%%%%%%%%%%%%%%%%%%%%%%%%%%%%%%%%%%%%%%%%%%%%%%%%%%%%%%%%%%%%%%%%%%%
\usepackage{amssymb}
\usepackage{eurosym}
\usepackage{amsmath}

\setcounter{MaxMatrixCols}{10}
%TCIDATA{OutputFilter=LATEX.DLL}
%TCIDATA{Version=5.00.0.2552}
%TCIDATA{<META NAME="SaveForMode" CONTENT="1">}
%TCIDATA{Created=Thursday, December 14, 2000 09:16:34}
%TCIDATA{LastRevised=Thursday, August 26, 2004 17:37:31}
%TCIDATA{<META NAME="GraphicsSave" CONTENT="32">}
%TCIDATA{<META NAME="DocumentShell" CONTENT="Articles\SW\Elbert Walker's">}
%TCIDATA{Language=American English}
%TCIDATA{CSTFile=LaTeX article (bright).cst}

\newtheorem{theorem}{Theorem}
\newtheorem{acknowledgement}[theorem]{Acknowledgement}
\newtheorem{algorithm}[theorem]{Algorithm}
\newtheorem{axiom}[theorem]{Axiom}
\newtheorem{case}[theorem]{Case}
\newtheorem{claim}[theorem]{Claim}
\newtheorem{conclusion}[theorem]{Conclusion}
\newtheorem{condition}[theorem]{Condition}
\newtheorem{conjecture}[theorem]{Conjecture}
\newtheorem{corollary}[theorem]{Corollary}
\newtheorem{criterion}[theorem]{Criterion}
\newtheorem{definition}[theorem]{Definition}
\newtheorem{example}[theorem]{Example}
\newtheorem{exercise}[theorem]{Exercise}
\newtheorem{lemma}[theorem]{Lemma}
\newtheorem{notation}[theorem]{Notation}
\newtheorem{problem}[theorem]{Problem}
\newtheorem{proposition}[theorem]{Proposition}
\newtheorem{remark}[theorem]{Remark}
\newtheorem{solution}[theorem]{Solution}
\newtheorem{summary}[theorem]{Summary}
\newenvironment{proof}[1][Proof]{\textbf{#1.} }{\ \rule{0.5em}{0.5em}}
\def\Pr{\mbox{Pr}}
\def\V{\mbox{V}}
\def\Cov{\mbox{Cov}}
\def\E{\mbox{E}}
\def\C{\mbox{C}}
\def\Var{\mbox{Var}}
\def\R{\underline{R}}
\def\x{\underline{x}}
\def\S{\underline{S}}
\def\1{\mbox{{\bf 1\kern-.24em{I}}}}
\def\R{R}
\def\SE{\mbox{SE}}
\def\Sk{\mbox{Sk}}
\def\Ku{\mbox{Ku}}
\def\Rc{{\cal R}}
\def\G#1#2{\Gamma\left(\frac{#1}{#2}\right)}
\addtolength{\oddsidemargin}{-1cm}
\addtolength{\evensidemargin}{-1cm}
\addtolength{\topmargin}{-2cm}
\addtolength{\textwidth}{2cm}
\addtolength{\textheight}{3cm}
\renewcommand{\baselinestretch}{1.4}
\input{tcilatex}

\begin{document}

\title{Note GMM estimation}
\author{Michael Rockinger}
\maketitle

\section{\protect\bigskip Introduction}

In this note, you may find a short description how the code \texttt{GMM.m}
operates. The code is still in its trial phase. If you find any instability,
please, let me know.

The syntax of this function is very similar to the one you have seen with
the Maxlik.m function. This is logic. GMM.m involves Maxlik.m

\section{Mathematical analysis}

We suppose that some statistical problem can be written as%
\begin{equation*}
\min_{\theta }g(\theta )\Omega ^{-1}g(\theta ).
\end{equation*}

Here, $g(\theta )$ is some function mapping a $R^{p}$ vector of parameters
into $R^{q}.$ Examples of such a function will be given below. It is assumed
that each component of the function $g(\theta )$ is associated with a
certain measure of precision, translated as a matrix of variance covariance.
The minimization then states that one seeks the parameters that minimize a
certain quadratic form where each component gets weighted by its relative
precision.

Example 1: It is possible to set OLS and IV estimation in that framework.
Define%
\begin{equation*}
y_{t}=x_{t}\beta +u_{t}.
\end{equation*}

We assume that we have for each observation a certain vector of instruments,
that is a vector of entities such that 
\begin{equation*}
E[z_{t}u_{t}]=0.
\end{equation*}

In many cases, the $x_{t}$ are just right. In that case, $\theta =\beta $
and $q=1.$ We may assume that there are more instruments than elements of $%
x_{t}.$ Asymptotically:

Example 2: Estimation of an Euler equation.

\section{Implementation}

[ beta, stderr, covbeta, Qmin, test, ptest ] = GMM(MomFct,beta0,...

A,b,Aeq,beq,lb,ub,nonlcon,options,...

GMMLags,GMMiter,GMMtol1,GMMtol2,varargin);

\bigskip

\end{document}
