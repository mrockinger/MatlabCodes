
\documentclass[12pt,a4paper]{article}
%%%%%%%%%%%%%%%%%%%%%%%%%%%%%%%%%%%%%%%%%%%%%%%%%%%%%%%%%%%%%%%%%%%%%%%%%%%%%%%%%%%%%%%%%%%%%%%%%%%%%%%%%%%%%%%%%%%%%%%%%%%%%%%%%%%%%%%%%%%%%%%%%%%%%%%%%%%%%%%%%%%%%%%%%%%%%%%%%%%%%%%%%%%%%%%%%%%%%%%%%%%%%%%%%%%%%%%%%%%%%%%%%%%%%%%%%%%%%%%%%%%%%%%%%%%%
\usepackage{eurosym}
\usepackage{amsmath}

\setcounter{MaxMatrixCols}{10}
%TCIDATA{OutputFilter=LATEX.DLL}
%TCIDATA{Version=5.00.0.2552}
%TCIDATA{<META NAME="SaveForMode" CONTENT="1">}
%TCIDATA{Created=Thursday, December 14, 2000 09:16:34}
%TCIDATA{LastRevised=Tuesday, March 23, 2004 18:30:43}
%TCIDATA{<META NAME="GraphicsSave" CONTENT="32">}
%TCIDATA{<META NAME="DocumentShell" CONTENT="Articles\SW\Elbert Walker's">}
%TCIDATA{Language=American English}
%TCIDATA{CSTFile=LaTeX article (bright).cst}

\newtheorem{theorem}{Theorem}
\newtheorem{acknowledgement}[theorem]{Acknowledgement}
\newtheorem{algorithm}[theorem]{Algorithm}
\newtheorem{axiom}[theorem]{Axiom}
\newtheorem{case}[theorem]{Case}
\newtheorem{claim}[theorem]{Claim}
\newtheorem{conclusion}[theorem]{Conclusion}
\newtheorem{condition}[theorem]{Condition}
\newtheorem{conjecture}[theorem]{Conjecture}
\newtheorem{corollary}[theorem]{Corollary}
\newtheorem{criterion}[theorem]{Criterion}
\newtheorem{definition}[theorem]{Definition}
\newtheorem{example}[theorem]{Example}
\newtheorem{exercise}[theorem]{Exercise}
\newtheorem{lemma}[theorem]{Lemma}
\newtheorem{notation}[theorem]{Notation}
\newtheorem{problem}[theorem]{Problem}
\newtheorem{proposition}[theorem]{Proposition}
\newtheorem{remark}[theorem]{Remark}
\newtheorem{solution}[theorem]{Solution}
\newtheorem{summary}[theorem]{Summary}
\newenvironment{proof}[1][Proof]{\textbf{#1.} }{\ \rule{0.5em}{0.5em}}
\def\Pr{\mbox{Pr}}
\def\V{\mbox{V}}
\def\Cov{\mbox{Cov}}
\def\E{\mbox{E}}
\def\C{\mbox{C}}
\def\Var{\mbox{Var}}
\def\R{\underline{R}}
\def\x{\underline{x}}
\def\S{\underline{S}}
\def\1{\mbox{{\bf 1\kern-.24em{I}}}}
\def\R{R}
\def\SE{\mbox{SE}}
\def\Sk{\mbox{Sk}}
\def\Ku{\mbox{Ku}}
\def\Rc{{\cal R}}
\def\G#1#2{\Gamma\left(\frac{#1}{#2}\right)}
\addtolength{\oddsidemargin}{-1cm}
\addtolength{\evensidemargin}{-1cm}
\addtolength{\topmargin}{-2cm}
\addtolength{\textwidth}{2cm}
\addtolength{\textheight}{3cm}
\renewcommand{\baselinestretch}{1.4}
\input{tcilatex}

\begin{document}

\title{FAME: Applied Econometrics\\
ML Syntax\ \ \ \ }
\author{Michael Rockinger}
\date{23.March.04}
\maketitle

\section{Description of the syntax of the maximum likelihood module}

[beta,stderr,vc,logl]=Max\_lik(lik\_fct,b0,vc%
\_type,A,b,Aeq,beq,lb,ub,nonlcon,options,varargin);

\section{Description}

This module performs the estimation of a parameter vector via maximum
likelihood. This estimation is largely built around the Matlab procedure
that computes constrained minimization named fmincon. Many of the parameters
need to be provided to that optimizer.

\section{Input parameters}

\textbf{lik\_fct} a function handle

\textbf{b0} the initial values

\textbf{vc\_type} instructions which variance-covariance matrix to
implement. Currently available procedures are Hessian and White's Sandwiched
method.

the next set of parameters is feed to the optimizer. We use the optimizer in
its most general way.

\subsection{fmincon}

The optimizer gets called with:

\begin{equation*}
\text{fmincon(fun,x0,A,b,Aeq,beq,lb,ub,nonlcon,options,P1,P2,...)}
\end{equation*}

fmincon finds the constrained minimum of a scalar function of several
variables starting at an initial estimate. This is generally referred to as
constrained nonlinear optimization or nonlinear programming.

fmincon starts at \textbf{x0} and finds a minimum \textbf{x} to the function
described in \textbf{fun} subject to the linear inequalities \textbf{A}*x 
\TEXTsymbol{<}= \textbf{b}. x0 can be a scalar, vector, or matrix. It is
possible to minimize \textbf{fun} subject to the linear equalities \textbf{%
Aeq}*x = \textbf{beq} as well as A*x \TEXTsymbol{<}= b. Set A=[] and b=[] if
no inequalities exist. Set Aeq=[] and beq=[] if no equalities exist.

\textbf{lb} and \textbf{ub} defines a set of lower and upper bounds on the
variables, x, so that the solution is always in the range lb \TEXTsymbol{<}=
x \TEXTsymbol{<}= ub. It is always possible to define lb and ub very large
so that the constrained optimization can also be used for unconstrained
optimization.

nonlcon subjects the minimization to the nonlinear inequalities \textbf{c(x)}
or equalities \textbf{ceq(x)} defined in \textbf{nonlcon}. fmincon optimizes
such that c(x) \TEXTsymbol{<}= 0 and ceq(x) = 0. \textbf{options} minimizes
with the optimization parameters specified in the structure \ `options'. 
\textbf{P1,P2,...} passes the problem-dependent parameters P1, P2, etc.,
directly to the functions fun and nonlcon. 

\bigskip 

[x,fval,exitflag,output,lambda,grad,hessian] = fmincon(...) returns 

\textbf{x} the minimum

\textbf{fval} the value of f evaluated at the minimum

\textbf{exitflag} a value that describes the exit condition of fmincon.

\textbf{output} a structure output with information about the optimization.

\textbf{lambda} is a structure lambda whose fields contain the Lagrange
multipliers at the solution x.

\textbf{grad} returns the value of the gradient of fun at the solution x.

\textbf{hessian} returns the value of the Hessian of fun at the solution x.

\subsection{fun}

the function to be optimized should have as a first argument the vector of
parameters over which the optimization should be done.

then the function may import a list of parameters. 

It is possible to include in this list of parameters the vector of data.
Typically this would be a vector containing $T$ observations. 

fun should export the vector of log-likelihoods.

\section{Output parameters}

\textbf{beta} vector of parameters that maximize the log\_likelihood

\textbf{stderr} vector of associated standard errors. These standard errors
get computed either via numerical hessians or via the sandwich method.

\textbf{vc} the variance covariance matricx. Again the same method as for
stderr is used 

\textbf{logl} the log-likelihood of the problem.

\end{document}
